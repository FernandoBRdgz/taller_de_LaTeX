\documentclass[12pt]{report}
\usepackage[spanish]{babel}
\usepackage[latin1]{inputenc}

\begin{document}

\title{FBRTL21}
\author{Fernando Barranco Rodr�guez}
\maketitle


Una de las principales problem�ticas de nuestro tiempo es la creciente complejidad de la estructura y funcionamiento de sistemas modernos, por lo que se requiere de nuevos m�todos formales apropiados para la especificaci�n de sistemas y para su validaci�n tanto cualitativa como cuantitativa. Lasredes de Petrison formalismos �tiles en el an�lisis de sistemas modernos ya que permiten describirlos de manera concisa y apropiada debido a que ofrecen una representaci�n gr�fica ordenada, vers�til y l�gica de los m�dulos que componena un sistema, as� como de la interacci�n que guardan �stos entre s�. Las redes de Petri fueron creadas en 1962 por Carl Adam Petri para su tesis doctoral, en la cual describe sus fundamentos te�ricos. Desde entonces, cada vez son m�s los estudiosos de este tema y han surgido muchas variedades de las mismas en los �ltimos a�os.

Cuando la aleatoriedad se manifiesta en un sistema la complejidad del mismo aumenta considerablemente, por lo que es necesario el uso de herramientas probabil�sticas que permitan su correcta modelaci�n y estudio. La teor�a de la probabilidad se inici� pr�cticamente con el an�lisis matem�tico de los juegos de azar realizado primero por los matem�ticos Pierre du Fermat (1601-1665) y Blas Pascal (1623-1662). Christian Huygens (1629-1695) public� en 1657 el primer tratados obre problemas relacionados con juegos de azar, el cual sirvi� como base para el gran desarrollo experimentado por esta teor�a durante el siglo XVIII, con el aporte de Jacobo Bernoulli (1654-1705) y Pierre Simon Laplace (1749-1827). Fueron Markov (1856-1922) y Kolmogorov (1903-1987) dos de los principales matem�ticos que contribuyeron al estudio de la aleatoriedad a trav�s de modelos conocidos comoprocesos estoc�sticos, los cuales permiten modelar una amplia gama de fen�menos, bajo un s�lido fundamento matem�tico.

\end{document}}