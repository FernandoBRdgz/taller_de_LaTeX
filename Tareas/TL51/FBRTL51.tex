% Plantilla para el taller de LaTeX

\documentclass[12pt]{article}
\usepackage[spanish]{babel}
\usepackage[latin1]{inputenc}
\usepackage{enumerate}

\begin{document}

\title{FBRTL51}
\author{Fernando Barranco Rodr�guez}
\date{9 Enero 2017}
\maketitle

\newpage
\noindent
a. \\
\linebreak
\noindent
Existen diversas estrategias para minimizar el error de redondeo (v�ase \cite{a}).
\begin{thebibliography}{XXX99}
\bibitem{a}
Akai, Terrence J. \emph{M�todos Num�ricos}. Limusa, M�xico, 2002.
\end{thebibliography}
b. \\
\linebreak
\noindent
Existen diversas estrategias para minimizar el error de redondeo (v�ase \cite{b}).
\begin{thebibliography}{XXX99}
\bibitem[A02]{b}
Akai, Terrence J. \emph{M�todos Num�ricos}. Limusa, M�xico, 2002.
\end{thebibliography}
d. \\
\linebreak
\noindent
Existen diversas estrategias para minimizar el error de redondeo (v�ase \cite{d}).
\begin{thebibliography}{99}
\bibitem[1]{d}
Akai, T. J. \emph{M�todos Num�ricos}. Limusa, M�xico, 2002.
\end{thebibliography}
c. \\
\linebreak
\noindent
Existen diversas estrategias para minimizar el error de redondeo 
\footnote{Akai, Terrence J. \emph{M�todos Num�ricos}. Limusa, M�xico, 2002.}.


\end{document}