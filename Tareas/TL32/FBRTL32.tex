% Plantilla para el taller de LaTeX

\documentclass[12pt]{article}
\usepackage[spanish]{babel}
\usepackage[latin1]{inputenc}
\usepackage{amsmath}

\begin{document}

\title{FBRTL32}
\author{Fernando Barranco Rodr�guez}
\date{6 Enero 2017}
\maketitle

\newpage

\begin{enumerate} 

\item \begin{align*}
h_{w_{H}} (\delta) &= \min_{z \in (0,1] } \log_{q} \frac{f_{w_H}}{z^\delta} \\
&= \min_{z \in (0,1] } \left( \log_{q} f_{w_{H}} (z)- \log_{q} z^\delta  \right) \\
&= \min_{z \in (0,1] } \left( \log_{q}\left( 1+ \left(q-1 \right) z \right )- \delta \log_{q}z \right) \\
&= \log_{q} \left( 1 + \left( q-1\right) \frac{\delta}{\left( q-1 \right) \left( 1-\delta \right)} \right)
   - \delta \log_{q} \left( \frac{\delta}{\left( q-1 \right) \left( 1-\delta \right)} \right) \\
&= \log_{q} \left( \frac{1}{1-\delta} \right) - \delta \log_{q} \delta + \delta \log_{q} \left( q-1 \right) 
   + \delta \log_{q} \left( q-1 \right) \\
&= \delta \log_{q} \frac{1}{\delta} + \left( 1-\delta \right) \log_{q} \frac{1}{1-\delta} + \delta \log_{q} \left( q-1 \right).
\end{align*}

\item \begin{align*}
ab &= [x_1, x_2]qx_2[x_1, x_2][x_1, x_2]x_1+q^{-1}qx_2[x_1, x_2] \left[ [x_1, x_2]+q^{-1}x_1x_2 \right] [x_1, x_2]x_1 \\
&= [x_1, x_2]qx_2[x_1, x_2][x_1, x_2]x_1 + x_2[x_1, x_2][x_1, x_2][x_1, x_2]x_1 \\
& \quad + x_2[x_1, x_2] \\
\end{align*}

\item \begin{align}
[x_{i}, x_{j}] &= 0, \qquad \textrm{si} \; |i-j| > 1 ; \\
[[x_{i}, x_{i+1}],x_{i+1}] &= 0, \qquad \textrm{si} \; 1 \leq i <n; \\
[x_{i}, [x_{i}, x_{i+1}]] &= 0, \qquad \textrm{si} \; 1 \leq i <n.
\end{align}

\item \begin{align}
\setcounter{equation}{0}
[x_{i}, x_{j}] &= 0, \qquad \textrm{si} \; |i-j| > 1 ; \nonumber \\
[[x_{i}, x_{i+1}],x_{i+1}] &= 0, \qquad \textrm{si} \; 1 \leq i <n; \nonumber \\
[x_{i}, [x_{i}, x_{i+1}]] &= 0, \qquad \textrm{si} \; 1 \leq i <n.
\end{align} 

\item \begin{align*}
e^{i\theta_{1}}e^{i\theta_{2}} &= ( \cos \theta_{1} + i\sin \theta_{1} )( \cos \theta_{2} + i\sin \theta_{2} ) \\
&= ( \cos \theta_{1}\cos \theta_{2} - \sin \theta_{1}\sin \theta_{2} ) + i( \cos \theta_{1}\sin \theta_{2} + \sin \theta_{1}
   \cos \theta_{2} ) \\
&= \cos( \theta_{1} + \theta_{2} ) + i\sin( \theta_{1} + \theta_{2} ) \\
&= e^{i(\theta_{1} + \theta_{2})}
\end{align*}

\end{enumerate}

\end{document}


