% Plantilla para el taller de LaTeX

\documentclass[12pt]{article}
\usepackage[spanish]{babel}
\usepackage[latin1]{inputenc}
\usepackage{graphicx}
\usepackage{enumerate}
\usepackage{float}


\begin{document}


\title{Ejemplo para insertar gr�ficos y figuras \\ en \LaTeXe}
\author{Fernando Barranco Rodr�guez}
\date{8 de Enero de 2017.}
\maketitle

Este es un peque�o ejemplo que muestra los efectos producidos por diversas opciones del entorno figure. 
Comencemos con el comando \texttt{/listoffigures} que produce el siguiente �ndice:

\listoffigures

Este es el gr�fico en su tama�o original:

\begin{center}
\includegraphics{LaTeXlion2.png}
\end{center}

\newpage

\begin{figure}[H]
\centering
\includegraphics[width=\textwidth]{LaTeXlion2.png}
\caption[Gr�fico con textwidth]{Gr�fico con textwidth}
\label{fig:1}
\end{figure}


\begin{figure}[H]
\centering
\includegraphics[width=1 cm]{LaTeXlion2.png}
\caption[Gr�fico con width=1.00 cm]{Gr�fico con width=1.00 cm}
\label{fig:2}
\end{figure}


\begin{figure}
\centering
\includegraphics[width=1 cm, height=3 cm]{LaTeXlion2.png}
\caption[Gr�fico con width=1.00 cm y height=3.00 cm]{Gr�fico con width=1.00 cm y height=3.00 cm}
\label{fig:3}
\end{figure}


\begin{figure}
\centering
\includegraphics[width=1.0 cm, height=3.0 cm, keepaspectratio]{LaTeXlion2.png}
\caption[Gr�fico con width=1.00 cm, height=3.00 cm y keepaspectratio]{Gr�fico con width=1.00 cm, height=3.00 cm y keepaspectratio}
\label{fig:4}
\end{figure}


\begin{figure}
\centering
\includegraphics[width=1.0 cm, height=1.0 cm, draft]{LaTeXlion2.png}
\caption[Gr�fico con width=1.00 cm, height=1.00 cm y draft]{Gr�fico con width=1.00 cm, height=1.00 cm y draft}
\label{fig:5}
\end{figure}


\begin{figure}
\centering
\includegraphics[scale=0.75]{LaTeXlion2.png}
\caption[Gr�fico con scale=0.75]{Gr�fico con scale=0.75}
\label{fig:6}
\end{figure}


\begin{figure}
\centering
\includegraphics[width=3.0 cm, height=3.0 cm, angle=45]{LaTeXlion2.png}
\caption[Gr�fico con width=3.00 cm, height=3.00 cm y angle=45]{Gr�fico con width=3.00 cm, height=3.00 cm y angle=45}
\label{fig:7}
\end{figure}


\newpage
\section*{Par�metros de \texttt{/includegraphics} }

\bigskip
\bigskip
\begin{enumerate}[a.]
\item \textbf{width:}\\ Indica a \LaTeX\ la Anchura de la im�gen. (debe especificarse preferentemente en cent�metros)
\item \textbf{height:}\\ Indica a \LaTeX\ la Altura de la im�gen. (debe especificarse preferentemente en cent�metros)
\item \textbf{textwidth:}\\ Indica a \LaTeX\ que debe Escalar la im�gen de forma ``ajustada".
\item \textbf{keepaspectratio:}\\ Indica a \LaTeX\ que Mantenga el aspecto en forma de cuadrado independientemente de si se indica una altura distinta a la anchura.
\item \textbf{scale:}\\ Indica a \LaTeX\ la Escala de la im�gen. (debe especificarse preferentemente en porcentaje)
\item \textbf{draft:}\\ Indica a \LaTeX\ que ``imprima" la Ruta donde se ubica guardada la im�gen.
\item \textbf{angle:}\\ Indica a \LaTeX\ el �ngulo de la im�gen. (debe especificarse preferentemente en grados)
\end{enumerate}

\end{document}