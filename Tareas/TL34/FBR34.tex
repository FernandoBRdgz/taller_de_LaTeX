% Plantilla para el taller de LaTeX

\documentclass[12pt]{article}
\usepackage[spanish]{babel}
\usepackage[latin1]{inputenc}
\usepackage{amsthm}


\swapnumbers
\begin{document}

\title{FBRTL34}
\author{Fernando Barranco Rodr�guez}
\date{7 Enero 2017}
\maketitle
\newpage

\section{Ejemplos de enunciados}

\theoremstyle{definition}
\newtheorem{def1}{Definici�n}[section]
\begin{def1}
Sea $u$ un entero positivo. Un c�digo $C$ se dice \textit{detector de $u$
errores} si, siempre que una palabra c�digo incurre en al menos un error y a
lo m�s $u$ errores, la palabra resultante no es una palabra c�digo. Un c�digo
$C$ se dice \textit{detector de exactamente $u$ errores} si es detector de $u$ errores, pero
no detector de $(u+1)$ errores. 
\end{def1}

\theoremstyle{remark}
\newtheorem{ej1}[def1]{Ejemplo}
\begin{ej1}
Sea el c�digo binario $C = \{00000, 00111,11222\}$. Analicemos 
los cambios necesarios en las coordenadas de cada palabra c�digo de manera
que podamos obtener alguna otra palabra c�digo existente en $C$.
\end{ej1}

\theoremstyle{plain}
\newtheorem{teo1}[def1]{Teorema}
\begin{teo1}
Un c�digo C es \emph{ detector de u errores } si y s�lo si $d(C) \geq u+1$ es decir, un c�digo con distancia d es un c�digo corrector de exactamente $(d-1)$ errores.
\proof
La prueba se sigue por definici�n.
\endproof
\end{teo1}


\end{document}
