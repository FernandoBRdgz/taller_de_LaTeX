% Plantilla para el taller de LaTeX

\documentclass[12pt]{article}
\usepackage[spanish]{babel}
\usepackage[latin1]{inputenc}
\usepackage[datebegin]{flexbib}

\begin{document}

\title{Bibliogrf�a con Flexbib \\ {\small FBRTL54} }
\author{Fernando Barranco Rodr�guez}
\date{9 de Enero de 2017}
\maketitle

La teor�a de c�digos es el resultado de una maravillosa combinaci�n de la
teor�a de c�digos correctores de erroes y matem�ticas para la modelaci�n de
la comunicaci�n confiable en la presencia de ruido, involucrando matem�ticas
discretas c�lculo combinatorio, �lgebra moderna, �lgebra lineal, teor�a de
probabilidad y estad�stica, La teor�a de c�digos ha sido investigada y desarrollada
durante m�s de cinco d�cadas y ha visto gran aplicaci�n en diversos
�mbitos que involucran la t�nsmici�n de informaci�n codificada (v�ase \cite{Lint}).
Mientras que originalmente la teor�a algebraica de c�digos correctores
de errores tuvo lugar en el escenario de los espacios vectoriales sobre campos
finitos, el estudio de los c�digos lineales sobre anillos finitos ha cobrado
fuerza e importancia a partir de que, a�os atr�s, especialistas en la materia
descubrieron que c�digos aparentemente no lineales en realidad est�n relacionados
con c�digos lineales sobre el anillo de los enteros m�dulo cuatro
(v�ase \cite{calderbank}).

\bibliographystyle{flexbib}
\bibliography{biblio_ejemplo}
\nocite{*}

\end{document}