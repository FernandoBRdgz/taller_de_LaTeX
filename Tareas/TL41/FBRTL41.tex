% Plantilla para el taller de LaTeX

\documentclass[12pt]{article}
\usepackage[spanish]{babel}
\usepackage[latin1]{inputenc}
\usepackage{booktabs}

\begin{document}

\title{Un Ejemplo con Excel2LaTeX \\ {\small FBRTL41} }
\author{Fernando Barranco Rodr�guez}
\date{7 Enero 2017}
\maketitle

\newpage

% Table generated by Excel2LaTeX from sheet 'COTAS ASINTOTICAS'
El Cuadro \ref{tab:cotas} es un peque�o ejemplo de lo que puede hacerse usando Excel2LaTeX.

\begin{table}[htbp]
  \centering
  \caption{Mi primera tabla usando Excel2LaTeX}
    \begin{tabular}{cccc}
    \toprule
    \toprule
    \boldmath{}\textbf{$\delta$}\unboldmath{} & \textbf{Cota de Gilbert-Varshamov} & \textbf{Cota de Hamming} & \textbf{Cota de Elias} \\
    \midrule
    0.04  & 0.88180 & 0.93414 & 0.93356 \\
    0.08  & 0.79108 & 0.88180 & 0.87975 \\
    0.12  & 0.71125 & 0.83470 & 0.83039 \\
    0.16  & 0.63884 & 0.79108 & 0.78376 \\
    0.20  & 0.57220 & 0.75009 & 0.73907 \\
    0.24  & 0.51040 & 0.71125 & 0.69583 \\
    0.28  & 0.45283 & 0.67424 & 0.65375 \\
    0.32  & 0.39909 & 0.63884 & 0.61260 \\
    0.36  & 0.34889 & 0.60487 & 0.57220 \\
    0.40  & 0.30204 & 0.57220 & 0.53243 \\
    0.44  & 0.25839 & 0.54074 & 0.49319 \\
    0.48  & 0.21787 & 0.51040 & 0.45437 \\
    0.52  & 0.18044 & 0.48111 & 0.41592 \\
    0.56  & 0.14610 & 0.45283 & 0.37774 \\
    0.60  & 0.11488 & 0.42550 & 0.33980 \\
    0.64  & 0.08687 & 0.39909 & 0.30204 \\
    0.68  & 0.06221 & 0.37356 & 0.26440 \\
    0.72  & 0.04108 & 0.34889 & 0.22687 \\
    0.76  & 0.02379 & 0.32505 & 0.18942 \\
    0.80  & 0.01073 & 0.30204 & 0.15206 \\
    0.84  & 0.00250 & 0.27982 & 0.11488 \\
    0.88  & 0.00006 & 0.25839 & 0.07811 \\
    0.92  & 0.00502 & 0.23775 & 0.04246 \\
    0.96  & 0.02088 & 0.21787 & 0.01073 \\
    \bottomrule
    \bottomrule
    \end{tabular}%
  \label{tab:cotas}%
\end{table}%


\end{document}